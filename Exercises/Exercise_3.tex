\documentclass[a4paper,12pt]{article}

\usepackage{amsmath}
\usepackage{graphicx}
\usepackage{tabularx}
\usepackage{xcolor}
\usepackage{listings}
\usepackage{ragged2e}


\renewcommand*{\thesection}{\hspace{-5mm}}
\renewcommand*{\thesubsection}{(\alph{subsection})}

\setlength{\parindent}{0em}

\definecolor{Red}{RGB}{255, 0, 0}
\definecolor{Blue}{RGB}{0, 0, 255}

\begin{document}
	
	\title{Mixed-Effects Models, Spring 2022}
	\author{Mathieu Simon}
	\maketitle
	
	\setlength{\parskip}{1em}
	
	\section{Exercise 1}
	
	\subsection{}
	
	As in the Example 2.20 of the lecture notes, we can make use of the fact that \textit{Block} (1\textsuperscript{st}-level random factor) is always observed with \textit{sample} and \textit{dilut} (2\textsuperscript{nd}-level random factor). Therefore we can define:\\
	
	$ \mathbf{Z}_i^* = 
	\begin{bmatrix}
		\mathbf{Z}_{br1} & \mathbf{0} & \mathbf{0} & \mathbf{0} & \mathbf{0} & \mathbf{0} \\
		\mathbf{0} & ... & \mathbf{0} & \mathbf{0} & \mathbf{0} & \mathbf{0}\\
		\mathbf{0} & \mathbf{0} & \mathbf{Z}_{br6} & \mathbf{0} & \mathbf{0} & \mathbf{0} \\
		\mathbf{0} & \mathbf{0} & \mathbf{0} & \mathbf{Z}_{bc1} & \mathbf{0} & \mathbf{0} \\
		\mathbf{0} & \mathbf{0} & \mathbf{0} & \mathbf{0} & ... & \mathbf{0} \\
		\mathbf{0} & \mathbf{0} & \mathbf{0} & \mathbf{0} & \mathbf{0} & \mathbf{Z}_{bc5} \\
	\end{bmatrix}, 
	\mathbf{b}_i^* =
	\begin{bmatrix}
		b_i + r_{i1} \\
		b_i + r_{i2} \\
		b_i + r_{i3} \\
		b_i + r_{i4} \\
		b_i + r_{i5} \\
		b_i + r_{i6} \\
		b_i + c_{i1} \\
		b_i + c_{i2} \\
		b_i + c_{i3} \\
		b_i + c_{i4} \\
		b_i + c_{i5} \\
	\end{bmatrix}, 
	\mathbf{\Psi}^* = 
	\begin{bmatrix}
		\sigma_1^2 + \sigma_2^2 \mathbf{I}_6 & \sigma_1^2 \\
		\sigma_1^2 & \sigma_1^2 + \sigma_3^2 \mathbf{I}_5
	\end{bmatrix}$
	
	\newpage
	\subsection{}
	\subsubsection*{assay.lme.2b}
	\begin{lstlisting}[language=R]
lme(logDens ~ sample*dilut, 
random = list(Block = pdBlocked(
list(pdCompSymm(~ dilut - 1), pdCompSymm(~ sample - 1)))),
data=Assay.df)
	\end{lstlisting}

	\subsection{}
	\subsubsection*{assay.lme.3}
	\begin{lstlisting}[language=R]
lme(logDens ~ sample * dilut, 
random = list(Block = pdBlocked(
list(pdIdent(~ 1), pdIdent(~ sample - 1), 
     pdIdent(~ dilut - 1)))),
weights = varIdent(form = ~ 1 | sample * dilut),
data=Assay.df)
	\end{lstlisting}

	The variance for \textit{sample} stays constant but differs for different \textit{dilut}.
	
	\section{Exercise 2}
	\subsection{}
	Step performed as in the example on p. 214–221 of Pinheiro \& Bates
	\subsubsection*{fm2Dial.lme}
	\begin{lstlisting}[language=R]
lme(rate ~(pressure + I(pressure^2) +
           I(pressure^3) + I(pressure^4))*QB,
random = ~ pressure + I(pressure^2) | Subject,
weights = varPower(form = ~ pressure) , data = Dialyzer.df)
	\end{lstlisting}
	
	\subsection{}
	Residual variance for pressure = 2 dmHg and flow rate = 200 dl/min:\\[1em]
	$\text{varPower : }Var\left(\epsilon_{ij}\right) \quad=\quad \sigma^2 |v_{ij}|^{2\delta} \quad\tilde{=}\quad 4.5$\\[1em]
	With: $\sigma^2 = 1.26^2$, $v_{ij} = 2$ (pressure), and $\delta = 0.749$\\[1em]
	
	NB: in the book, \textit{fm3Dial.lme} allows for different variance according to the flow rate which is not the case for \textit{fm2Dial.lme}.
	
	\subsection{}
	\subsubsection*{fm4Dial.lme}
	\begin{lstlisting}[language=R]
lme(rate ~(pressure + I(pressure^2) +
           I(pressure^3) + I(pressure^4))*QB,
random = ~ pressure + I(pressure^2) | Subject,
weights = varExp(form = ~ pressure) , data = Dialyzer.df)
	\end{lstlisting}

	\section{Exercise 3}
	
	
	\section{Exercise 4}

	
	
	
\end{document}
