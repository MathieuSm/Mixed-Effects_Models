\documentclass[a4paper,12pt]{article}

\usepackage{amsmath}
\usepackage{graphicx}
\usepackage{tabularx}
\usepackage{xcolor}
\usepackage{listings}
\usepackage{ragged2e}
\usepackage{pifont}



\renewcommand*{\thesection}{\hspace{-5mm}}
\renewcommand*{\thesubsection}{(\alph{subsection})}

\setlength{\parindent}{0em}

\definecolor{Red}{RGB}{255, 0, 0}
\definecolor{Blue}{RGB}{0, 0, 255}

\begin{document}
	
	\title{Mixed-Effects Models, Spring 2022}
	\author{Mathieu Simon}
	\maketitle
	
	\setlength{\parskip}{1em}
	
	\section{Exercise 1}
	
	\subsection{}
	
	As in the Example 2.20 of the lecture notes, we can make use of the fact that \textit{Block} (1\textsuperscript{st}-level random factor) is always observed with \textit{sample} and \textit{dilut} (2\textsuperscript{nd}-level random factor). Therefore we can define:\\
	
	$ \mathbf{Z}_i^* = 
	\begin{bmatrix}
		\mathbf{Z}_{br1} & \mathbf{0} & \mathbf{0} & \mathbf{0} & \mathbf{0} & \mathbf{0} \\
		\mathbf{0} & ... & \mathbf{0} & \mathbf{0} & \mathbf{0} & \mathbf{0}\\
		\mathbf{0} & \mathbf{0} & \mathbf{Z}_{br6} & \mathbf{0} & \mathbf{0} & \mathbf{0} \\
		\mathbf{0} & \mathbf{0} & \mathbf{0} & \mathbf{Z}_{bc1} & \mathbf{0} & \mathbf{0} \\
		\mathbf{0} & \mathbf{0} & \mathbf{0} & \mathbf{0} & ... & \mathbf{0} \\
		\mathbf{0} & \mathbf{0} & \mathbf{0} & \mathbf{0} & \mathbf{0} & \mathbf{Z}_{bc5} \\
	\end{bmatrix}, 
	\mathbf{b}_i^* =
	\begin{bmatrix}
		b_i + r_{i1} \\
		b_i + r_{i2} \\
		b_i + r_{i3} \\
		b_i + r_{i4} \\
		b_i + r_{i5} \\
		b_i + r_{i6} \\
		b_i + c_{i1} \\
		b_i + c_{i2} \\
		b_i + c_{i3} \\
		b_i + c_{i4} \\
		b_i + c_{i5} \\
	\end{bmatrix}, 
	\mathbf{\Psi}^* = 
	\begin{bmatrix}
		\sigma_1^2 + \sigma_2^2 \mathbf{I}_6 & \sigma_1^2 \\
		\sigma_1^2 & \sigma_1^2 + \sigma_3^2 \mathbf{I}_5
	\end{bmatrix}$
	
	\newpage
	\subsection{}
	\subsubsection*{assay.lme.2b}
	\begin{lstlisting}[language=R]
lme(logDens ~ sample*dilut, 
random = list(Block = pdBlocked(
list(pdCompSymm(~ dilut - 1), pdCompSymm(~ sample - 1)))),
data=Assay.df)
	\end{lstlisting}

	\subsection{}
	\subsubsection*{assay.lme.3}
	\begin{lstlisting}[language=R]
lme(logDens ~ sample * dilut, 
random = list(Block = pdBlocked(
list(pdIdent(~ 1), pdIdent(~ sample - 1), 
     pdIdent(~ dilut - 1)))),
weights = varIdent(form = ~ 1 | sample * dilut),
data=Assay.df)
	\end{lstlisting}

	The variance for \textit{sample} stays constant but differs for different \textit{dilut}.
	
	\section{Exercise 2}
	\subsection{}
	Step performed as in the example on p. 214–221 of Pinheiro \& Bates
	\subsubsection*{fm2Dial.lme}
	\begin{lstlisting}[language=R]
lme(rate ~(pressure + I(pressure^2) +
           I(pressure^3) + I(pressure^4))*QB,
random = ~ pressure + I(pressure^2) | Subject,
weights = varPower(form = ~ pressure) , data = Dialyzer.df)
	\end{lstlisting}
	
	\subsection{}
	Residual variance for pressure = 2 dmHg and flow rate = 200 dl/min:\\[1em]
	$\text{varPower : }Var\left(\epsilon_{ij}\right) \quad=\quad \sigma^2 |v_{ij}|^{2\delta} \quad\tilde{=}\quad 4.5$\\[1em]
	With: $\sigma^2 = 1.26^2$, $v_{ij} = 2$ (pressure), and $\delta = 0.749$\\[1em]
	
	NB: in the book, \textit{fm3Dial.lme} allows for different variance according to the flow rate which is not the case for \textit{fm2Dial.lme}.
	
	\subsection{}
	\subsubsection*{fm4Dial.lme}
	\begin{lstlisting}[language=R]
lme(rate ~(pressure + I(pressure^2) +
           I(pressure^3) + I(pressure^4))*QB,
random = ~ pressure + I(pressure^2) | Subject,
weights = varExp(form = ~ pressure) , data = Dialyzer.df)
	\end{lstlisting}

	\section{Exercise 3}
	\subsection{}
	\subsubsection*{Oats.lme5}
	\begin{lstlisting}[language=R]
lme(yield ~ nitro, random = ~1|Block,
    corr = corSymm(form = ~1|Block/Variety), data=Oats.df)
	\end{lstlisting}
	
	\newpage
	\subsection{}
	\subsubsection*{intervals(Oats.lme5)}
	\begin{lstlisting}[language=R]
Correlation structure:
             lower      est.     upper
cor(1,2) -0.04975240 0.4503464 0.7698921
cor(1,3) -0.28284208 0.2501623 0.6651202
cor(1,4) -0.04795947 0.4457068 0.7643755
cor(2,3) -0.16370072 0.3096073 0.6670499
cor(2,4)  0.27208984 0.6330247 0.8378083
cor(3,4) -0.02484203 0.4082015 0.7122568
	\end{lstlisting}
	
	\subsubsection*{Oats.lme6}
	\begin{lstlisting}[language=R]
lme(yield ~ nitro, random = ~1|Block,
corr = corCompSymm(form = ~1|Block/Variety), data=Oats.df)
	\end{lstlisting}
	
	\subsubsection*{anova(Oats.lme5,Oats.lme6)}
	\begin{lstlisting}[language=R]
Model df    AIC    BIC  logLik  Test L.Ratio p-value
    1 10 610.18 632.66 -295.09                        
    2  5 603.04 614.28 -296.52  1vs2   2.856   0.722
	\end{lstlisting}
	
	\newpage
	\subsection{}
	\subsubsection*{Oats.lme4}
	\begin{lstlisting}[language=R]
Log-restricted-likelihood: -296.52

Fixed: yield ~ nitro
(Intercept)   nitro 
  81.8722    73.6666

Random effects:
Formula: ~1 | Block
         (Intercept)
StdDev:    14.5059

Formula: ~1 | Variety %in% Block
         (Intercept) Residual
StdDev:    11.0046   12.86696
	\end{lstlisting}
	
	\subsubsection*{Oats.lme6}
	\begin{lstlisting}[language=R]
Log-restricted-likelihood: -296.52

Fixed: yield ~ nitro
(Intercept)   nitro 
  81.8722    73.6666

Random effects:
Formula: ~1 | Block
         (Intercept) Residual
StdDev:    14.5059   16.93109

Correlation Structure: Compound symmetry
Formula: ~1 | Block/Variety 
Parameter estimate(s):
   Rho 
0.4224608
	\end{lstlisting}
	
	\newpage
	Verifications:\\[1em]
	$ 16.931^2 \quad\tilde{=}\quad 11.005^2 + 12.867^2 $ \ding{51}\\
	$ 0.422 \quad\tilde{=}\quad 11.005^2 / (11.005^2 + 12.867^2) $ \ding{51}
	
	\subsection{}
	\subsubsection*{Oats.gls1}
	\begin{lstlisting}[language=R]
gls(yield ~ nitro,
correlation = corCompSymm(form = ~1 | Block/Variety),
data=Oats.df)
	\end{lstlisting}
	
	\subsubsection*{anova(Oats.lme6,Oats.gls1)}
	\begin{lstlisting}[language=R]
Model df  AIC    BIC    logLik  Test L.Ratio p-value
    1  5 603.04 614.28 -296.52                        
    2  4 606.06 615.05 -299.03  1vs2  5.023   0.025
	\end{lstlisting}
	The \textit{Block effect} seems to be significant as the \textit{p-value} is lower than 0.05.
	
	\newpage
	\section{Exercise 4}

	\subsubsection*{fm3BW.lme}
	\begin{lstlisting}[language=R]
lme( weight ~ Time * Diet, random = ~ Time | Rat,
weights = varPower(), corr = corExp(form = ~ Time),
data = BodyWeight.df)
	\end{lstlisting}
	
	\subsubsection*{fm3BW.gls}
	\begin{lstlisting}[language=R]
gls( weight ~ Time * Diet, weights = varPower(),
corr = corCAR1(form = ~ Time | Rat), data = BodyWeight.df)
	\end{lstlisting}

	\subsubsection*{anova(fm3BW.lme,fm3BW.gls)}
	\begin{lstlisting}[language=R]
Model df  AIC    BIC    logLik  Test L.Ratio p-value
    1 12 1145.1 1182.8 -560.570                        
    2  9 1147.7 1175.9 -564.843 1vs2  8.546   0.036
	\end{lstlisting}
	According to the AIC value the more general model (\textit{fm3BW.lme}) seems to be more appropriate but the BIC value indicate the opposite. The two models are not nested because, even if they have the same fixed effects, they are not fitted the same way. This is was we can not use the p-value (cannot perform L ratio test) to compare the models.
	
	
\end{document}
